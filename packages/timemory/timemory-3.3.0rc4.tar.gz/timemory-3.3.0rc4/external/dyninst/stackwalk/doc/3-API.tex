\section{API Reference}
\label{sec:api}

This section describes the StackwalkerAPI interface. It is divided into three
sub-sections: a description of the definitions and basic types used by this API,
a description of the interface for collecting stackwalks, and a description of
the callback interface.

\subsection{Definitions and Basic Types}
The following definitions and basic types are referenced throughout the rest of
this manual.

\subsubsection{Definitions}
\label{subsec:definitions}
\begin{description}

\item [Stack Frame] A stack frame is a record of a function (or function-like
    object) invocation. When a function is executed, it may create a frame on
    the call stack. StackwalkerAPI finds stack frames and returns a description
    of them when it walks a call stack.  The following three definitions deal
    with stack frames.

\item [Bottom of the Stack] The bottom of the stack is the earliest stack frame
    in a call stack, usually a thread's initial function. The stack grows from
    bottom to the top.

\item [Top of the Stack] The top of the stack is the most recent stack frame in
    a call stack. The stack frame at the top of the stack is for the currently
    executing function.

\item [Frame Object] A Frame object is StackwalkerAPI's representation of a
    stack frame. A Frame object is a snapshot of a stack frame at a specific
    point in time. Even if a stack frame changes as a process executes, a Frame
    object will remain the same. Each Frame object is represented by an instance
    of the Frame class.

\end{description}

The following three definitions deal with fields in a Frame object. 
\begin{description}

\item [SP (Stack Pointer)] A Frame object's SP member points to the top of its
    stack frame (a stack frame grows from bottom to top, similar to a call
    stack). The Frame object for the top of the stack has a SP that is equal to
    the value in the stack pointer register at the time the Frame object was
    created. The Frame object for any other stack frame has a SP that is equal
    to the top address in the stack frame. 

\item [FP (Frame Pointer)] A Frame object's FP member points to the beginning
    (or bottom) of its stack frame. The Frame object for the top of the stack
    has a FP that is equal to the value in the frame pointer register at the
    time the Frame object was created. The Frame object for any other stack
    frame has a FP that is equal to the beginning of the stack frame.
    
\item [RA (Return Address)] A Frame object's RA member points to the location in
    the code space where control will resume when the function that created the
    stack frame resumes. The Frame object for the top of the stack has a RA that
    is equal to the value in the program counter register at the time the Frame
    object was created.  The Frame object for any other stack frame has a RA
    that is found when walking a call stack.

\end{description}

\begin{figure}
\centering
\tikzstyle{class} = [rectangle, draw, minimum width=3cm, minimum height=1.5em,
font=\ttfamily, node distance=1.5em]
\tikzstyle{top} = [rectangle, minimum width=3cm, draw, minimum height=1.5em,
fill=white, draw=white]
\tikzstyle{annotate} = [rectangle]
\tikzstyle{code} = [rectangle, font=\ttfamily]
\tikzstyle{line} = [draw, -latex']
\begin{tikzpicture}[
        every text node part/.style={align=left}
    ]

    % Label
    \node [annotate] (callstack) {\textbf{Call Stack}};

    % The callstack
    \node [top, below of=callstack] (dots) {...};
    \node [class, below of=dots] (a) {a};
    \node [class, below of=a] (b) {b};
    \node [class, below of=b] (mainRA) {main's RA};
    \node [class, below of=mainRA] (mainFP) {main's FP};
    \node [class, below of=mainFP] (c) {C};
    \node [class, below of=c] (fooRA) {foo's RA};
    \node [class, below of=fooRA] (fooFP) {foo's FP};
    \node [class, below of=fooFP] (d) {d};

    % Registers
    \node [annotate, below=1cm of d] (registers) {Registers};
    \node [class, below of=registers] (FramePointer) {Frame Pointer};
    \node [class, below of=FramePointer] (ProgramCounter) {Program Counter};
    \node [class, below of=ProgramCounter] (StackPointer) {StackPointer};

    % Left labels
    \node [code, left=6cm of dots,anchor=north] (main) {void main() \{\\
        \hspace{1cm}int a;\\
        \hspace{1cm}foo(0);\\
        \hspace{1cm}...\\
        \}
    };
    
    \node [code, below=3cm of main.west,anchor=west] (foo) {void foo(int b) \{\\
        \hspace{1cm}int c;\\
        \hspace{1cm}bar();\\
        \hspace{1cm}...\\
        \}
    };

    \node [code, below=3cm of foo.west,anchor=west] (bar) {void bar() \{\\
        \hspace{1cm}int d;\\
        \hspace{1cm}while(1);\\
    \}
    };

    % Callstack->callstack labels
    \path [line] (mainFP.east) [bend right=20] to (dots.east);
    \path [line] (fooFP.east) [bend right=40] to (mainFP.east);
    \path [line] (FramePointer.east) [bend right=40] to (fooFP.east);
    \path [line] (StackPointer.east) [bend right=40] to (d.east);

    % Right labels
    \node [code, left=2.5cm of dots,anchor=south] (mainFO) {main\textnormal{'s}\\ Frame Object};
    \node [code, below=.25em of mainFO] (mainFPlabel) {FP};
    \path [line] (mainFPlabel) -- (a.west);
    \node [code, below=.25em of mainFPlabel] (mainRAlabel) {RA};
    \path [line, dashed] (mainRAlabel) -- (mainRA.west);
    \node [code, below=.25em of mainRAlabel] (mainSPlabel) {SP};
    \path [line] (mainSPlabel) -- (mainFP.west);

    \node [code, below=2.5cm of mainFO] (fooFO) {foo\textnormal{'s}\\ Frame Object};
    \node [code, below=.25em of fooFO] (fooFPlabel) {FP};
    \path [line] (fooFPlabel) -- (c.west);
    \node [code, below=.25em of fooFPlabel] (fooRAlabel) {RA};
    \path [line, dashed] (fooRAlabel) -- (fooRA.west);
    \node [code, below=.25em of fooRAlabel] (fooSPlabel) {SP};
    \path [line] (fooSPlabel) -- (fooFP.west);

    \node [code, below=2.5cm of fooFO] (barFO) {bar\textnormal{'s}\\ Frame Object};
    \node [code, below=.25em of barFO] (barFPlabel) {FP};
    \path [line, dashed] (barFPlabel) -- (FramePointer.west);
    \node [code, below=.25em of barFPlabel] (barRAlabel) {RA};
    \path [line, dashed] (barRAlabel) -- (ProgramCounter.west);
    \node [code, below=.25em of barRAlabel] (barSPlabel) {SP};
    \path [line, dashed] (barSPlabel) -- (StackPointer.west);

    % Callstack->source code labels
    \path [line] (mainRAlabel.west) -- ($(main.south east) + (-.75cm, 2em)$);
    \path [line] (fooRAlabel.west) -- ($(foo.south east) + (-1.5cm, 2em)$);
    \path [line] (barRAlabel.west) -- ($(bar.south east) + (0, 2em)$);


\end{tikzpicture}

\begin{tikzpicture}[]
    % Legend
    \node [annotate] (la1) {A};
    \node [annotate, right of=la1] (lb1) {B};
    \path [line] (la1) -- (lb1);
    \node [annotate, right=1cm of lb1] (l1) {A contains B's address};
    
    \node [annotate, below=.5cm of la1] (la2) {A};
    \node [annotate, right of=la2] (lb2) {B};
    \path [line, dashed] (la2) -- (lb2);
    \node [annotate, right=1cm of lb2] (l2) {A contains the contents of B};

    % Legend Box
    \node [rectangle, draw, fit=(la1)(lb1)(l1)(la2)(lb2)(l2)] (legendBox) {};
    \node [rectangle, label=left: \rotatebox{90}{LEGEND},fit=(legendBox)] {};
\end{tikzpicture}

\caption{Stack Frame and Frame Object Layout (x86 Architecture)}
\label{fig:layout}

\end{figure}



\begin{figure}
\centering
\tikzstyle{class} = [rectangle, draw, minimum width=4cm, minimum height=1.5em,
font=\ttfamily, node distance=1.5em]
\tikzstyle{top} = [rectangle, minimum width=4cm, draw, minimum height=1.5em,
fill=white, draw=white]
\tikzstyle{annotate} = [rectangle]
\tikzstyle{code} = [rectangle, font=\ttfamily]
\tikzstyle{line} = [draw, -latex']
\begin{tikzpicture}[
        every text node part/.style={align=left}
    ]

    % Label
    \node [annotate] (callstack) {\textbf{Call Stack}};

    % The callstack
    \node [top, below of=callstack] (dots) {...};
    \node [class, below of=dots] (a) {a};
    \node [class, below of=a] (b) {b};
    \node [class, below of=b] (mainRA) {X's RA(0x400000c)};
\node [class, below of=mainRA] (mainFP) {X's FP};
    \node [class, below of=mainFP] (c) {c};
    \node [class, below of=c] (fooRA) {main's RA(0x400010c)};
    \node [class, below of=fooRA] (fooFP) {main's FP};
    \node [class, below of=fooFP] (d) {d};
    \node [class, below of=d] (barRA) {foo's RA(0x400020c)};
    \node [class, below of=barRA] (barFP) {foo's FP};

    % Registers
    \node [annotate, below=1cm of barFP] (registers) {Registers};
    \node [class, below of=registers] (FramePointer) {Frame Pointer};
    \node [class, below of=FramePointer] (StackPointer) {StackPointer};
    \node [class, below of=StackPointer] (ProgramCounter) {Program Counter};

    % Left labels
    \node [code, left=7.5cm of dots,anchor=north] (main) {
        0x4000100\hspace{0.5cm}void main() \{\\
        0x4000104\hspace{1cm}int a;\\
        0x4000108\hspace{1cm}foo(0);\\
        0x400010c\hspace{1cm}...\\
        \hspace{2.2cm}\}
    };

    \node [code, above=2.5cm of main.west, anchor=west](callmain){
        0x4000000\hspace{0.5cm}\_\_X:\\
        0x400000x\hspace{1cm}...\\
        0x4000008\hspace{1cm}jmp main\\
        0x400000c\hspace{1cm}...\\
    };

    \node [code, above=1.5cm of callmain.west, anchor=west](codeTitle){
        pseudo line number\\and code\\
    };
    
    \node [code, below=3cm of main.west,anchor=west] (foo) {
        0x4000200\hspace{0.5cm}void foo(int b) \{\\
        0x4000204\hspace{1cm}int c;\\
        0x4000208\hspace{1cm}bar();\\
        0x400020c\hspace{1cm}...\\
        \hspace{2.2cm}\}
    };

    \node [code, below=3cm of foo.west,anchor=west] (bar) {
        0x4000300\hspace{0.5cm}void bar() \{\\
        0x4000304\hspace{1cm}int d;\\
        0x4000308\hspace{1cm}while(1);\\
        \hspace{2.2cm}\}
    };

    % Callstack->callstack labels
    \path [line] (mainFP.east) [bend right=20] to (dots.east);
    \path [line] (fooFP.east) [bend right=40] to (mainFP.east);
    \path [line] (barFP.east) [bend right=40] to (fooFP.east);
    \path [line] (FramePointer.east) [bend right=40] to (barFP.east);
    \path [line] (StackPointer.east) [bend right=40] to (barFP.east);

    % Right labels
    \node [code, left=2.5cm of dots,anchor=south] (mainFO) {main\textnormal{'s}\\ Frame Object};
    \node [code, below=.25em of mainFO] (mainFPlabel) {FP};
    \path [line] (mainFPlabel) -- (mainFP.west);
    \node [code, below=.25em of mainFPlabel] (mainSPlabel) {SP};
    \path [line] (mainSPlabel) -- (mainFP.west);
    \node [code, below=.25em of mainSPlabel] (mainRAlabel) {RA};
    \path [line, dashed] (mainRAlabel) -- (fooRA.west);

    \node [code, below=2cm of mainFO] (fooFO) {foo\textnormal{'s}\\ Frame Object};
    \node [code, below=.25em of fooFO] (fooFPlabel) {FP};
    \path [line] (fooFPlabel) -- (fooFP.west);
    \node [code, below=.25em of fooFPlabel] (fooSPlabel) {SP};
    \path [line] (fooSPlabel) -- (fooFP.west);
    \node [code, below=.25em of fooSPlabel] (fooRAlabel) {RA};
    \path [line, dashed] (fooRAlabel) -- (barRA.west);

    \node [code, below=2cm of fooFO] (barFO) {bar\textnormal{'s}\\ Frame Object};
    \node [code, below=.25em of barFO] (barFPlabel) {FP};
    \path [line, dashed] (barFPlabel) -- (FramePointer.west);
    \path [line] (barFPlabel) -- (barFP.west);
    \node [code, below=.25em of barFPlabel] (barSPlabel) {SP};
    \path [line, dashed] (barSPlabel) -- (StackPointer.west);
    \path [line] (barSPlabel) -- (barFP.west);
    \node [code, below=.25em of barSPlabel] (barRAlabel) {RA};
    \path [line, dashed] (barRAlabel) -- (ProgramCounter.west);

    % Callstack->source code labels
    \path [line] (mainRAlabel.west) -- ($(main.south east) + (-.75cm, 2em)$);
    \path [line] (fooRAlabel.west) -- ($(foo.south east) + (-1.5cm, 2em)$);
    \path [line] (barRAlabel.west) -- ($(bar.south east) + (0, 2em)$);


\end{tikzpicture}

\begin{tikzpicture}[]
    % Legend
    \node [annotate] (la1) {A};
    \node [annotate, right of=la1] (lb1) {B};
    \path [line] (la1) -- (lb1);
    \node [annotate, right=1cm of lb1] (l1) {A contains B's address};
    
    \node [annotate, below=.5cm of la1] (la2) {A};
    \node [annotate, right of=la2] (lb2) {B};
    \path [line, dashed] (la2) -- (lb2);
    \node [annotate, right=1cm of lb2] (l2) {A contains the contents of B};

    % Legend Box
    \node [rectangle, draw, fit=(la1)(lb1)(l1)(la2)(lb2)(l2)] (legendBox) {};
    \node [rectangle, label=left: \rotatebox{90}{LEGEND},fit=(legendBox)] {};
\end{tikzpicture}

\caption{Stack Frame and Frame Object Layout (ARMv8 Architecture)}
\label{fig:layout-armv8}

\end{figure}




Figure~\ref{fig:layout} shows the relationship between application code, stack
frames, and Frame objects. In the figure, the source code on the left has run
through the main and foo functions, and into the bar function. It has created
the call stack in the center, which is shown as a sequence of words growing
down. The current values of the processor registers, while executing in bar, are
shown below the call stack. When StackwalkerAPI walks the call stack, it creates
the Frame objects shown on the right. Each Frame object corresponds to one of
the stack frames found in the call stack or application registers.  

The call stack in Figure~\ref{fig:layout} is similar to one that would be found
on the x86 architecture. Details about how the call stack is laid out may be
different on other architectures, but the meanings of the FP, SP, and RA fields
in the Frame objects will remain the same. The layout of the ARM64 stack may be found in Figure~\ref{fig:layout-armv8} as an example of the scope of architectural variations.


The following four definitions deal with processes involved in StackwalkerAPI.
\begin{description}

\item [Target Process] The process from which StackwalkerAPI is collecting
    stackwalks.

\item [Host Process] The process in which StackwalkerAPI code is currently
    running.

\item [First Party Stackwalk] StackwalkerAPI collects first party stackwalk when
    it walks a call stack in the same address space it is running in, i.e. the
    target process is the same as the host process.

\item [Third Party Stackwalk] StackwalkerAPI collects third party stackwalk when
    it walks the call stack in a different address space from the one it is
    running in, i.e. the target process is different from the host process. A
    third party stackwalk is usually done through a debugger interface.

\end{description}

\subsubsection{Basic Types}

\begin{apient}
typedef unsigned long Address
\end{apient}
\apidesc{
    An integer value capable of holding an address in the target process.
    Address variables should not, and in many cases cannot, be used directly as
    a pointer. It may refer to an address in a different process, and it may not
    directly match the target process' pointer representation. Address is
    guaranteed to be at least large enough to hold an address in a target
    process, but may be larger.
}
\begin{apient}
typedef ... Dyninst::PID
\end{apient}
\apidesc{
	A handle for identifying a process. On UNIX systems this will be an integer representing a PID. On Windows this will be a HANDLE object.
}
\begin{apient}
typedef ... Dyninst::THR_ID
\end{apient}
\apidesc{
	A handle for identifying a thread. On Linux platforms this is an integer referring to a TID (Thread Identifier). On Windows it is a HANDLE object.
}

\begin{apient}
class Dyninst::MachRegister
\end{apient}
\apidesc{
	A value that names a machine register.
}

\begin{apient}
typedef unsigned long Dyninst::MachRegisterVal
\end{apient}
\apidesc{
	A value that holds the contents of a register. A Dyninst::MachRegister names a specific register, while a Dyninst::MachRegisterVal represents the value that may be in that register.
}    
    
\subsection{Namespace StackwalkerAPI}
The classes in Section~\ref{sec:stackwalking-interface} and
Section~\ref{sec:callback-interface} fall under the C++ namespace
Dyninst::Stackwalker. To access them, a user should refer to them using the
Dyninst::Stackwalker:: prefix, e.g. Dyninst::Stackwalker::Walker. Alternatively,
a user can add the C++ using keyword above any references to StackwalkerAPI
objects, e.g, using namespace Dyninst and using namespace Stackwalker.
    
\subsection{Stackwalking Interface}
\label{sec:stackwalking-interface}

This section describes StackwalkerAPI's interface for walking a call stack. This
interface is sufficient for walking call stacks on all the systems and
variations covered by our default callbacks. 

To collect a stackwalk, first create new Walker object associated with the target process via

\begin{lstlisting}
    Walker::newWalker()
\end{lstlisting}
	or 
\begin{lstlisting}
    Walker::newWalker(Dyninst::PID pid)
\end{lstlisting}

Once a Walker object has been created, a call stack can be walked with the
\begin{lstlisting}
Walker::walkStack
\end{lstlisting}
method. The new stack walk is returned as a vector of Frame objects.
    
\subsubsection{Class Walker}
\label{subsec:walker}
\definedin{walker.h}

The \code{Walker} class allows users to walk call stacks and query basic information
about threads in a target process. The user should create a \code{Walker} object for
each process from which they are walking call stacks. Each \code{Walker} object is
associated with one process, but may walk call stacks on multiple threads within
that process. The \code{Walker} class allows users to query for the threads available
for walking, and it allows you to specify a particular thread whose call stack
should be walked. Stackwalks are returned as a vector of Frame objects. 

Each Walker object contains three objects: 
\begin{itemize}
    \item ProcessState
    \item StepperGroup
    \item SymbolLookup
\end{itemize}

These objects are part of the Callback Interface and can be used to customize
StackwalkerAPI. The \code{ProcessState} object tells \code{Walker} how to access data in the
target process, and it determines whether this \code{Walker} collects first party or
third party stackwalks. \code{Walker} will pick an appropriate default \code{ProcessState}
object based on which factory method the users calls. The \code{StepperGroup} object is
used to customize how the \code{Walker} steps through stack frames. The
\code{SymbolLookup}
object is used to customize how StackwalkerAPI looks up symbolic names of the
function or object that created a stack frame.

\begin{apient}
static Walker *newWalker()
static Walker *newWalker(Dyninst::PID pid)
static Walker *newWalker(Dyninst::PID pid, std::string executable)
static Walker *newWalker(Dyninst::ProcControlAPI::Process::ptr proc);
static Walker *newWalker(std::string executable,
                         const std::vector<std::string> &argv)
static Walker *newWalker(ProcessState *proc,
                         StepperGroup *steppergroup = NULL ,
                         SymbolLookup *lookup = NULL)
\end{apient}
\apidesc{
These factory methods return new Walker objects:
\begin{itemize}
\item The first takes no arguments and returns a first-party stackwalker.
\item The second takes a PID representing a running process and returns a third-party stackwalker on that
process.
\item The third takes the name of the executing binary in addition to the PID and also returns a third-party
stackwalker on that process.
\item The fourth takes a ProcControlAPI process object and returns a third-party stackwalker.
\item The fifth takes the name of an executable and its arguments, creates the process, and returns a
third-party stackwalker.
\item The sixth takes a ProcessState pointer representing a running process as well as user-defined
StepperGroup and SymbolLookup pointers. It can return both first-party and third-party Walkers,
depending on the ProcessState parameter.
\end{itemize}

Unless overriden with the sixth variant, the new Walker object uses the default StepperGroup and
SymbolLookup callbacks for the current platform. First-party walkers use the ProcSelf callback
for its ProcessState object. Third-party walkers use ProcDebug instead. See Section 3.5.1 for
more information about defaults in the Callback Interface.

This method returns NULL if it was unable to create a new Walker object. The new Walker object
was created with the new operator, and should be deallocated with the delete operator when it
is no longer needed.
}
 
\begin{apient}
static bool newWalker(const std::vector<Dyninst::PID> &pids,
                      std::vector<Walker *> &walkers_out)
static bool newWalker(const std::vector<Dyninst::PID> &pids,
                      std::vector<Walker *> &walkers_out,
                      std::string executable)
\end{apient}
\apidesc{
 This method attaches to a group of processes and returns a vector of Walker objects that perform
third-party stackwalks. As above, the first variant takes a list of PIDs and attaches to those
processes; the second variant also specifies the executable binary.}

\begin{apient}
bool walkStack(std::vector<Frame> &stackwalk,	 
               Dyninst::THR_ID thread = NULL_THR_ID)
\end{apient}
\apidesc{
    This method walks a call stack in the process associated with this \code{Walker}.
    The call stack is returned as a vector of \code{Frame} objects in stackwalk. The
    top of the stack is returned in index 0 of stackwalk, and the bottom of the
    stack is returned in index \code{stackwalk.size()-1}.

    A stackwalk can be taken on a specific thread by passing a value
    in the thread parameter. If \code{thread} has the value
    \code{NULL\_THR\_ID}, then a default thread will be chosen. When
    doing a third party stackwalk, the default thread will be the
    process' initial thread. When doing a first party stackwalk, the
    default thread will be the thread that called
    \code{walkStack}. The default StepperGroup provided to a Walker
    will support collecting call stacks from almost all types of
    functions, including signal handlers and optimized, frameless
    functions.

    This method returns \code{true} on success and \code{false} on failure.
}
 
\begin{apient}
bool walkStackFromFrame(std::vector<Frame> &stackwalk, const Frame &frame)
\end{apient}
\apidesc{
    This method walks a call stack starting from the given stack frame,
    \code{frame}.
    The call stack will be output in the \code{stackwalk} vector, with frame stored in
    index 0 of \code{stackwalk} and the bottom of the stack stored in index
    \code{stackwalk.size()-1}.

    This method returns \code{true} on success and \code{false} on failure.
}
 
\begin{apient}
bool walkSingleFrame(const Frame &in, Frame &out)
\end{apient}
\apidesc{
    This methods walks through single frame, \code{in}. Parameter \code{out}
    will be set to \code{in}'s caller frame.

    This method returns \code{true} on success and \code{false} on failure.
}
 
\begin{apient}
bool getInitialFrame(Frame &frame, Dyninst::THR_ID thread = NULL_THR_ID)
\end{apient}
\apidesc{
    This method returns the \code{Frame} object on the top of the stack in parameter
    frame. Under \code{walkStack}, \code{frame} would be the one returned in index 0 of the
    \code{stackwalk} vector.  A stack frame can be found on a specific thread by
    passing a value in the thread parameter. If \code{thread} has the value
    \code{NULL\_THR\_ID}, then a default thread will be chosen. When doing a third party
    stackwalk, the default thread will be the process' initial thread. When
    doing a first party stackwalk, the default thread will be the thread that
    called \code{getInitialFrame}. 

    This method returns \code{true} on success and \code{false} on failure.
}
 
\begin{apient}
bool getAvailableThreads(std::vector<Dyninst::THR_ID> &threads)
\end{apient}
\apidesc{
    This method returns a vector of threads in the target process upon which
    StackwalkerAPI can walk call stacks. The threads are returned in output
    parameter \code{threads}. Note that this method may return a subset of the actual
    threads in the process. For example, when walking call stacks on the current
    process, it is only legal to walk the call stack on the currently running
    thread. In this case, \code{getAvailableThreads} returns a vector containing only
    the current thread.

    This method returns \code{true} on success and \code{false} on failure.
}
 
\begin{apient}
ProcessState *getProcessState() const
\end{apient}
\apidesc{
    This method returns the \code{ProcessState} object associated with this \code{Walker}.
}
 
\begin{apient}
StepperGroup *getStepperGroup() const
\end{apient}
\apidesc{
    This method returns the \code{StepperGroup} object associated with this \code{Walker}.
}
 
\begin{apient}
SymbolLookup *getSymbolLookup() const
\end{apient}
\apidesc{
    This method returns the \code{SymbolLookup} object associated with this \code{Walker}.
}

\begin{apient}
bool addStepper(FrameStepper *stepper)
\end{apient}
\apidesc{
This method adds a provided FrameStepper to those used by the Walker.
}

\begin{apient}
static SymbolReaderFactory *getSymbolReader()
\end{apient}
\apidesc{This method returns a factory for creating process-specific symbol readers. Unlike the above
methods it is global across all Walkers and is thus defined static.}

\begin{apient}
static void setSymbolReader(SymbolReaderFactory *);                       
\end{apient}
\apidesc{Set the symbol reader factory used when creating
  \code{Walker} objects.}

\begin{apient}
static void version(int &major, int &minor, int &maintenance)
\end{apient}

\apidesc{This method returns version information (e.g., 8, 0, 0 for
  the 8.0 release).}

\subsubsection{Class Frame}
\label{subsec:frame}
\definedin{frame.h}
	
The \code{Walker} class returns a call stack as a vector of \code{Frame} objects. As described
in Section~\ref{subsec:definitions}, each Frame object represents a stack frame, and contains a
return address (RA), stack pointer (SP) and frame pointer (FP). For each of
these values, optionally, it stores the location where the values were found.
Each Frame object may also be augmented with symbol information giving a
function name (or a symbolic name, in the case of non-functions) for the object
that created the stack frame.

The Frame class provides a set of functions (getRALocation, getSPLocation and
getFPLocation) that return the location in the target process' memory or
registers where the RA, SP, or FP were found. These functions may be used to
modify the stack. For example, the DyninstAPI uses these functions to change
return addresses on the stack when it relocates code. The RA, SP, and FP may be
found in a register or in a memory address on a call stack. 

\begin{apient}
static Frame *newFrame(Dyninst::MachRegisterVal ra,
                       Dyninst::MachRegisterVal sp, 
                       Dyninst::MachRegisterVal fp, 
                       Walker *walker)
\end{apient}
\apidesc{
    This method creates a new \code{Frame} object and sets the mandatory data
    members: RA, SP and FP. The new \code{Frame} object is associated with
    \code{walker}.
	
    The optional location fields can be set by the methods below.
	
    The new \code{Frame} object is created with the \code{new} operator, and the
    user should be deallocate it with the \code{delete} operator when it is no longer needed.
}

\begin{apient}
bool operator==(const Frame &)
\end{apient}
\apidesc{\code{Frame} objects have a defined equality operator.}


\begin{apient}
Dyninst::MachRegisterVal getRA() const
\end{apient}
\apidesc{
	This method returns this \code{Frame} object's return address.
}

\begin{apient}
void setRA(Dyninst::MachRegisterVal val)
\end{apient}
\apidesc{
    This method sets this \code{Frame} object's return address to \code{val}.
}

\begin{apient}
Dyninst::MachRegisterVal getSP() const
\end{apient}
\apidesc{
	This method returns this \code{Frame} object's stack pointer.
}

\begin{apient}
void setSP(Dyninst::MachRegisterVal val)
\end{apient}
\apidesc{
    This method sets this \code{Frame} object's stack pointer to \code{val}.
}

\begin{apient}
Dyninst::MachRegisterVal getFP() const
\end{apient}
\apidesc{
	This method returns this \code{Frame} object's frame pointer.
}

\begin{apient}
void setFP(Dyninst::MachRegisterVal val)
\end{apient}
\apidesc{
    This method sets this \code{Frame} object's frame pointer to \code{val}.
}

\begin{apient}
bool isTopFrame() const;                                                      
bool isBottomFrame() const;                                                   
\end{apient}
\apidesc{
  These methods return whether a \code{Frame} object is the top (e.g.,
  most recently executing) or bottom of the stack walk. }

\begin{apient}
typedef enum { 
    loc_address, 
    loc_register, 
    loc_unknown 
} storage_t;
\end{apient}

\begin{apient}
typedef struct {
    union {
        Dyninst::Address addr;
        Dyninst::MachRegister reg;
    } val;
    storage_t location;
} location_t; 
\end{apient}
\apidesc{
    The \code{location\_t} structure is used by the \code{getRALocation},
    \code{getSPLocation}, and \code{getFPLocation} methods to describe where in
    the process a \code{Frame} object's RA, SP, or FP were found. When walking a
    call stack these values may be found in registers or memory. If they were
    found in memory, the \code{location} field of \code{location\_t} will contain
    \code{loc\_address} and the \code{addr} field will contain the address where it was found.
    If they were found in a register the \code{location} field of \code{location\_t}
    will contain \code{loc\_register} and the \code{reg} field will refer to the register where
    it was found. If this \code{Frame} object was not created by a stackwalk
    (using the \code{newframe} factory method, for example), and has not had a set
    location method called, then location will contain \code{loc\_unknown}.
}

\begin{apient}
location_t getRALocation() const
\end{apient}
\apidesc{
	This method returns a \code{location\_t} describing where the RA was found.
}

\begin{apient}
void setRALocation(location_t newval)
\end{apient}
\apidesc{
	This method sets the location of where the RA was found to newval.
}

\begin{apient}
location_t getSPLocation() const
\end{apient}
\apidesc{
	This method returns a \code{location\_t} describing where the SP was found.
}

\begin{apient}
void setSPLocation(location_t newval)
\end{apient}
\apidesc{
    This method sets the location of where the SP was found to \code{newval}.
}

\begin{apient}
location_t getFPLocation() const
\end{apient}
\apidesc{
	This method returns a \code{location\_t} describing where the FP was found.
}

\begin{apient}
void setFPLocation(location_t newval)
\end{apient}
\apidesc{
    This method sets the location of where the FP was found to \code{newval}.
}

\begin{apient}
bool getName(std::string &str) const
\end{apient}
\apidesc{
This method returns a stack frame's symbolic name. Most stack frames are created
by functions, or function-like objects such as signal handlers or system calls.
This method returns the name of the object that created this stack frame. For
stack frames created by functions, this symbolic name will be the function name.
A symbolic name may not always be available for all \code{Frame} objects, such
as in cases of stripped binaries or special stack frames types.
	
The function name is obtained by using this \code{Frame} object's RA to call the
\code{SymbolLookup} callback. By default StackwalkerAPI will attempt to use the
\code{SymtabAPI} package to look up symbol names in binaries. If
\code{SymtabAPI} is not found, and no alternative \code{SymbolLookup} object is
present, then this method will return an error.
	
This method returns \code{true} on success and \code{false} on error.  
}

\begin{apient}
bool getObject(void* &obj) const
\end{apient}
\apidesc{
    In addition to returning a symbolic name (see \code{getName}) the \code{SymbolLookup}
    interface allows for an opaque object, a \code{void*}, to be associated with a
    \code{Frame} object. The contents of this \code{void*} is determined by the
\code{SymbolLookup} implementation. Under the default implementation that uses
SymtabAPI, the \code{void*} points to a Symbol object or NULL if no symbol is found.

This method returns \code{true} on success and \code{false} on error.  
}

\begin{apient}
  Walker *getWalker() const; 
\end{apient}

\apidesc{This method returns the \code{Walker} object that constructed
  this stack frame. }

\begin{apient}
  THR_ID getThread() const;                                                     
\end{apient}
\apidesc{This method returns the execution thread that the current
  \code{Frame} represents. }

\begin{apient}
FrameStepper* getStepper() const
\end{apient}
\apidesc{
    This method returns the \code{FrameStepper} object that was used to
    construct this \code{Frame} object in the \code{stepper} output parameter. 

    This method returns \code{true} on success and \code{false} on error.  
}

\begin{apient}
bool getLibOffset(std::string &lib, Dyninst::Offset &offset, void* &symtab) const
\end{apient}
\apidesc{
This method returns the DSO (a library or executable) and an offset into that
DSO that points to the location within that DSO where this frame was created.
\code{lib} is the path to the library that was loaded, and \code{offset} is the offset into
that library. The return value of the \code{symtab} parameter is dependent on the
SymbolLookup implementation-by default it will contain a pointer to a
Dyninst::Symtab object for this DSO. See the SymtabAPI Programmer's Guide for
more information on using Dyninst::Symtab objects.
}

\begin{apient}
bool nonCall() const
\end{apient}
\apidesc{
    This method returns whether a \code{Frame} object represents a function
    call; if \code{false}, the \code{Frame} may represent instrumentation, a
    signal handler, or something else.
}


\subsection{Mapping Addresses to Libraries}
\definedin{procstate.h}

StackwalkerAPI provides an interface to access the addresses where libraries are mapped in the
target process.

\begin{apient}
typedef std::pair<std::string, Address> LibAddrPair;
\end{apient}
\apidesc{
A pair consisting of a library filename and its base address in the target process.
}

\begin{apient}
class LibraryState
\end{apient}
\apidesc{
Class providing interfaces for library tracking. Only the public query interfaces below are user-facing; the other public
methods are callbacks that allow StackwalkerAPI to update its internal state.
}

\begin{apient}
   virtual bool getLibraryAtAddr(Address addr, LibAddrPair &lib) = 0;
\end{apient}
\apidesc{
Given an address \code{addr} in the target process, returns \code{true} and sets \code{lib} to the name and base address of the library containing
addr. Given an address outside the target process, returns \code{false}.
}
\begin{apient}
   virtual bool getLibraries(std::vector<LibAddrPair> &libs, bool allow\_refresh = true) = 0;
\end{apient}
\apidesc{
Fills \code{libs} with the libraries loaded in the target process. If \code{allow\_refresh} is true, this method will attempt to ensure
that this list is freshly updated via inspection of the process; if it is false, it will return a cached list.
}
\begin{apient}
   virtual bool getLibc(LibAddrPair &lc);
\end{apient}
\apidesc{
Convenience function to find the name and base address of the standard C runtime, if present.
}
\begin{apient}
   virtual bool getLibthread(LibAddrPair &lt);
\end{apient}
\apidesc{
Convenience function to find the name and base address of the standard thread library, if present (e.g. pthreads).
}
\begin{apient}
   virtual bool getAOut(LibAddrPair &ao) = 0;
\end{apient}
\apidesc{
Convenience function to find the name and base address of the executable.
}
\subsection{Accessing Local Variables}
\definedin{local\_var.h}
	
StackwalkerAPI can be used to access local variables found in the frames of a
call stack. The StackwalkerAPI interface for accessing the values of local
variables is closely tied to the SymtabAPI interface for collecting information
about local variables--SymtabAPI handles for functions, local variables, and
types are part of this interface. 

Given an initial handle to a SymtabAPI Function object, SymtabAPI can look up
local variables contained in that function and the types of those local
variables. See the SymtabAPI Programmer's Guide for more information.

\begin{apient}
static Dyninst::SymtabAPI::Function *getFunctionForFrame(Frame f)
\end{apient}
\apidesc{
This method returns a SymtabAPI function handle for the function that created the call stack frame, f. 
}

\begin{apient}
static int glvv_Success = 0;
static int glvv_EParam = -1;
static int glvv_EOutOfScope = -2;
static int glvv_EBufferSize = -3;
static int glvv_EUnknown = -4;
\end{apient}

\begin{apient}
static int getLocalVariableValue(Dyninst::SymtabAPI::localVar *var,
                                 std::vector<Frame> &swalk,
                                 unsigned frame,
                                 void *out_buffer,
                                 unsigned out_buffer_size)
\end{apient}
\apidesc{
    Given a local variable and a stack frame from a call stack, this function
    returns the value of the variable in that frame. The local variable is
    specified by the SymtabAPI variable object, \code{var}. \code{swalk} is a
    call stack that was collected via StackwalkerAPI, and \code{frame} specifies
    an index into that call stack that contains the local variable. The value of
    the variable is stored in \code{out\_buffer} and the size of
    \code{out\_buffer} should be specified in \code{out\_buffer\_size}.
	
    A local variable only has a limited scope with-in a target process'
    execution. StackwalkerAPI cannot guarantee that it can collect the correct
    return value of a local variable from a call stack if the target process is
    continued after the call stack is collected.
	
    Finding and collecting the values of local variables is dependent on
    debugging information being present in a target process' binary. Not all
    binaries contain debugging information, and in some cases, such as for
    binaries built with high compiler optimization levels, that debugging
    information may be incorrect.

    \code{getLocalVariableValue} will return on of the following values:
    \begin{description}
    
        \item [glvv\_Success] getLocalVariableValue was able to correctly read
            the value of the given variable.
   
        \item [glvv\_EParam] An error occurred, an incorrect parameter was
            specified (frame was larger than \code{swalk.size()}, or var was not a
            variable in the function specified by frame).
    
        \item [glvv\_EOutOfScope] An error occurred, the specified variable
            exists in the function but isn't live at the current execution
            point.
    
        \item [glvv\_EBufferSize] An error occurred, the variable's value does
            not fit inside \code{out\_buffer}. 
    
        \item [glvv\_EUnknown] An unknown error occurred. It is most likely that
            the local variable was optimized away or debugging information about
            the variable was incorrect.
    
    \end{description} 
}

\subsection{Callback Interface}
\label{sec:callback-interface}
This subsection describes the Callback Interface for StackwalkerAPI. The
Callback Interface is primarily used to port StackwalkerAPI to new platforms,
extend support for new types of stack frames, or integrate StackwalkerAPI into
existing tools.

The classes in this subsection are interfaces, they cannot be instantiated.  To
create a new implementation of one of these interfaces, create a new class that
inherits from the callback class and implement the necessary methods. To use a
new ProcessState, StepperGroup, or SymbolLookup class with StackwalkerAPI,
create a new instance of the class and register it with a new Walker object
using the
\begin{lstlisting}
Walker::newWalker(ProcessState *, StepperGroup *, SymbolLookup *)
\end{lstlisting}	
factory method (see Section~\ref{subsec:walker}). To use a new FrameStepper class with
StackwalkerAPI, create a new instance of the class and register it with a
StepperGroup using the
\begin{lstlisting}
StepperGroup::addStepper(FrameStepper *)
\end{lstlisting}
method (see Section~\ref{subsec:steppergroup}).

Some of the classes in the Callback Interface have methods with default
implementations. A new class that inherits from a Callback Interface can
optionally implement these methods, but it is not required. If a method requires
implementation, it is written as a C++ pure virtual method (\code{virtual funcName() =
0}). A method with a default implementation is written as a C++ virtual method
(\code{virtual funcName()}).

\subsubsection{Default Implementations}
\label{subsec:defaults}

The classes described in the Callback Interface are C++ abstract classes, or
interfaces. They cannot be instantiated. For each of these classes
StackwalkerAPI provides one or more default implementations on each platform.
These default implementations are classes that inherit from the abstract classes
described in the Callback Interface. If a user creates a Walker object without
providing their own \code{FrameStepper}, \code{ProcessState}, and
\code{SymbolLookup} objects, then StackwalkerAPI will use the default
implementations listed in Table~\ref{table:defaults}. These
implementations are described in Section \ref{sec:framesteppers}.

\begin{table}
\begin{tabular}{| l | l | l | l | l |}
    \hline
                    &   StepperGroup    & ProcessState      &   SymbolLookup    &   FrameStepper\\
    \hline
    Linux/x86       &   1. AddrRange    &   1. ProcSelf     &   1. SwkSymtab    &   1. FrameFuncStepper\\
    Linux/x86-64    &                   &   2. ProcDebug    &                   &   2. SigHandlerStepper\\
                    &                   &                   &                   &   3. DebugStepper\\
                    &                   &                   &                   &   4. AnalysisStepper\\ 
                    &                   &                   &                   &   5. StepperWanderer\\
                    &                   &                   &                   &   6. BottomOfStackStepper\\
   \hline
    Linux/PPC       &   1. AddrRange    &   1. ProcSelf     &   1. SwkSymtab    & 1. FrameFuncStepper\\
    Linux/PPC-64    &                   &   2. ProcDebug    &                   & 2. SigHandlerStepper\\
                    &                   &                   &                   & 3. AnalysisStepper\\
    \hline
    Windows/x86     &   1. AddrRange    &   1. ProcSelf     &   1. SwkSymtab    & 1. FrameFuncStepper\\
                    &                   &   2. ProcDebug    &                   & 2. AnalysisStepper \\
                    &                   &                   &                   & 3. StepperWanderer\\
                    &                   &                   &                   & 4. BottomOfStackStepper\\
    \hline
\end{tabular}
\caption{Callback Interface Defaults}
\label{table:defaults}
\end{table}

\subsubsection{Class FrameStepper}\label{subsec:framestepper}
\definedin{framestepper.h}
	
The \code{FrameStepper} class is an interface that tells StackwalkerAPI how to walk
through a specific type of stack frame. There may be many different ways of
walking through a stack frame on a platform, e.g, on Linux/x86 there are
different mechanisms for walking through system calls, signal handlers, regular
functions, and frameless functions. A single \code{FrameStepper} describes how to walk
through one of these types of stack frames.

A user can create their own \code{FrameStepper} classes that tell StackwalkerAPI how to
walk through new types of stack frames. A new \code{FrameStepper} object must be added
to a \code{StepperGroup} before it can be used. 

In addition to walking through individual stack frames, a \code{FrameStepper} tells its
\code{StepperGroup} when it can be used. The \code{FrameStepper} registers address ranges that
cover objects in the target process' code space (such as functions). These
address ranges should contain the objects that will create stack frames through
which the \code{FrameStepper} can walk. If multiple \code{FrameStepper} objects have
overlapping address ranges, then a priority value is used to determine which
\code{FrameStepper} should be attempted first.

\code{FrameStepper} is an interface class; it cannot be instantiated. Users who want to
develop new \code{FrameStepper} objects should inherit from this class and implement
the the desired virtual functions. The \code{getCallerFrame,
  getPriority}, and \code{getName} functions must be implemented; all
others may be overridden if desired. 

\begin{apient}
typedef enum { 
    gcf_success,
    gcf_stackbottom,
    gcf_not_me, 
    gcf_error 
} gcframe_ret_t
\end{apient}

\begin{apient}
virtual gcframe_ret_t getCallerFrame(const Frame &in, Frame &out) = 0
\end{apient}
\apidesc{
This method walks through a single stack frame and generates a Frame object that
represents the caller's stack frame. Parameter in will be a Frame object that
this FrameStepper is capable of walking through. Parameter out is an output
parameter that this method should set to the Frame object that called in.

There may be multiple ways of walking through a different types of stack frames.
Each \code{FrameStepper} class should be able to walk through a type of stack frame.
For example, on x86 one \code{FrameStepper} could be used to walk through stack frames
generated by ABI-compliant functions; out's FP and RA are found by reading from
in's FP, and out's SP is set to the word below in's FP. A different \code{FrameStepper}
might be used to walk through stack frames created by functions that have
optimized away their FP. In this case, in may have a FP that does not point
out's FP and RA. The \code{FrameStepper} will need to use other mechanisms to discover
out's FP or RA; perhaps the \code{FrameStepper} searches through the stack for the RA
or performs analysis on the function that created the stack frame.

If \code{getCallerFrame} successfully walks through in, it is required to set the
following parameters in out. See Section~\ref{subsec:frame} for more details on the values
that can be set in a Frame object:

\begin{description}
    \item [Return Address (RA)] The RA should be set with the
        \code{Frame::setRA} method.
    \item [Stack Pointer (SP)] The SP should be set with the \code{Frame::setSP} method.
    \item [Frame Pointer (FP)] The FP should be set with the \code{Frame::setFP} method
\end{description}

Optionally, getCallerFrame can also set any of following parameters in out:
\begin{description}
    \item [Return Address Location (RALocation)] The RALocation should be set
        with the \code{Frame::setRALocation()} method.
    \item [Stack Pointer Location (SPLocation)] The SPLocation should be set
        with the \code{Frame::setRALocation()} method.
    \item [Frame Pointer Location (FPLocation)] The FPLocation should be set
        with the \code{Frame::setFPLocation()} method.
\end{description}

If a location field in out is not set, then the appropriate
\code{Frame::getRALocation}, \code{Frame::getSPLocation} or
\code{Frame::getFPLocation} method will
return \code{loc\_unknown}.

\code{getCallerFrame} should return \code{gcf\_success} if it successfully walks
through in and creates an \code{out} \code{Frame} object. It should return
\code{gcf\_stackbottom} if in is the bottom of the stack and there are no stack
frames below it. It should return \code{gcf\_not\_me} if in is not the correct
type of stack frame for this \code{FrameStepper} to walk through. StackwalkerAPI
will then attempt to locate another \code{FrameStepper} to handle \code{in} or
abort the stackwalk. It should return \code{gcf\_error} if there was an error
and the stack walk should be aborted.
}

\begin{apient}
virtual void registerStepperGroup(StepperGroup *steppergroup)
\end{apient}
\apidesc{
This method is used to notify a \code{FrameStepper} when StackwalkerAPI adds it to a
\code{StepperGroup}. The \code{StepperGroup} to which this \code{FrameStepper} is being added is
passed in parameter steppergroup. This method can be used to initialize the
\code{FrameStepper} (in addition to any \code{FrameStepper} constructor).
}

\begin{apient}
virtual unsigned getPriority() const = 0
\end{apient}
\apidesc{
This method is used by the \code{StepperGroup} to decide which \code{FrameStepper} to use if
multiple \code{FrameStepper} objects are registered over the same address range (see
addAddressRanges in Section~\ref{subsec:steppergroup} for more information about address ranges).
This method returns an integer representing a priority level, the lower the
number the higher the priority.

The default \code{FrameStepper} objects provided by StackwalkerAPI all return
priorities between \code{0x1000} and \code{0x2000}. If two \code{FrameStepper} objects have an
overlapping address range, and they have the same priority, then the order in
which they are used is undefined.
}

\begin{apient}
FrameStepper(Walker *w);
\end{apient}
\apidesc{Constructor definition for all \code{FrameStepper}
  instances.}

\begin{apient}
virtual ProcessState *getProcessState();
\end{apient}
\apidesc{Return the \code{ProcessState} used by the
  \code{FrameStepper}. Can be overridden if the user desires.}

\begin{apient}
virtual Walker *getWalker();
\end{apient}

\apidesc{Return the \code{Walker} associated with the
  \code{FrameStepper}. Can be overridden if the user desires.}

\begin{apient}
typedef std::pair<std::string, Address> LibAddrPair;
typedef enum { library_load, library_unload } lib_change_t;
virtual void newLibraryNotification(LibAddrPair *libAddr, 
                                    lib_change_t change);
\end{apient}

\apidesc{This function is called when a new library is loaded by the
process; it should be implemented if the \code{FrameStepper} requires
such information.}

\begin{apient}
virtual const char *getName() const = 0;
\end{apient}

\apidesc{Returns a name for the \code{FrameStepper}; must be
  implemented by the user.}

\subsubsection{Class StepperGroup}
\label{subsec:steppergroup}
\definedin{steppergroup.h}

The \code{StepperGroup} class contains a collection of \code{FrameStepper} objects. The
\code{StepperGroup}'s primary job is to decide which \code{FrameStepper} should be used to
walk through a stack frame given a return address. The default \code{StepperGroup}
keeps a set of address ranges for each \code{FrameStepper}. If multiple \code{FrameStepper}
objects overlap an address, then the default \code{StepperGroup} will use a priority
system to decide.

\code{StepperGroup} provides both an interface and a default implementation of that
interface. Users who want to customize the \code{StepperGroup} should inherit from this
class and re-implement any of the below virtual functions.

\begin{apient}
StepperGroup(Walker *walker)
\end{apient}
\apidesc{
    This factory constructor creates a new \code{StepperGroup} object associated
with \code{walker}. 
}

\begin{apient}
virtual bool addStepper(FrameStepper *stepper)
\end{apient}
\apidesc{
    This method adds a new \code{FrameStepper} to this \code{StepperGroup}. The
    newly added stepper will be tracked by this \code{StepperGroup}, and it will
    be considered for use when walking through stack frames. 

    This method returns \code{\code{true}} if it successfully added the
    \code{FrameStepper}, and \code{\code{false}} on error.
}

\begin{apient}
virtual bool addStepper(FrameStepper *stepper, Address start, Address
end) = 0;
\end{apient}
\apidesc{Add the specified \code{FrameStepper} to the list of known
  steppers, and register it to handle frames in the range
  [\code{start}, \code{end}).}

\begin{apient}
virtual void registerStepper(FrameStepper *stepper);
\end{apient}

\apidesc{Add the specified \code{FrameStepper} to the list of known
  steppers and use it over the entire address space.}

\begin{apient}
virtual bool findStepperForAddr(Address addr, FrameStepper* &out, 
                                const FrameStepper *last_tried = NULL) = 0
\end{apient}
\apidesc{
    Given an address that points into a function (or function-like object),
    addr, this method decides which \code{FrameStepper} should be used to walk through
    the stack frame created by the function at that address. A pointer to the
    \code{FrameStepper} will be returned in parameter \code{out}. 

    It may be possible that the \code{FrameStepper} this method decides on is unable to
    walk through the stack frame (it returns \code{gcf\_not\_me} from
    \code{FrameStepper::getCallerFrame}). In this case StackwalkerAPI will call
    findStepperForAddr again with the last\_tried parameter set to the failed
    \code{FrameStepper}. findStepperForAddr should then find another \code{FrameStepper} to
    use. Parameter \code{last\_tried} will be set to NULL the first time getStepperToUse
    is called for a stack frame.

    The default version of this method uses address ranges to decide which
    \code{FrameStepper} to use. The address ranges are contained within the process'
    code space, and map a piece of the code space to a \code{FrameStepper} that can
    walk through stack frames created in that code range. If multiple
    \code{FrameStepper} objects share the same range, then the one with the highest
    priority will be tried first.
	
    This method returns \code{true} on success and \code{false} on failure.  
}

\begin{apient}
typedef std::pair<std::string, Address> LibAddrPair;
typedef enum { library_load, library_unload } lib_change_t;    
virtual void newLibraryNotification(LibAddrPair *libaddr, lib_change_t
change);
\end{apient}
\apidesc{Called by the StackwalkerAPI when a new library is loaded.}

\begin{apient}
Walker *getWalker() const
\end{apient}
\apidesc{
	This method returns the Walker object that associated with this StepperGroup.
}

\begin{apient}
void getSteppers(std::set<FrameStepper *> &);
\end{apient}
\apidesc{Fill in the provided set with all \code{FrameSteppers}
  registered in the \code{StepperGroup}. }


\subsubsection{Class ProcessState}
\label{subsec:processstate}
\definedin{procstate.h}

The ProcessState class is a virtual class that defines an interface through
which StackwalkerAPI can access the target process. It allows access to
registers and memory, and provides basic information about the threads in the
target process. StackwalkerAPI provides two default types of \code{ProcessState}
objects: \code{ProcSelf} does a first party stackwalk, and \code{ProcDebug} does a third party
stackwalk.

A new \code{ProcessState} class can be created by inheriting from this class and
implementing the necessary methods. 

\begin{apient}
static ProcessState *getProcessStateByPid(Dyninst::PID pid)
\end{apient}
\apidesc{Given a \code{PID}, return the corresponding
  \code{ProcessState} object.}


\begin{apient}
virtual unsigned getAddressWidth() = 0;
\end{apient}
\apidesc{Return the number of bytes in a pointer for the target
  process. This value is 4 for 32-bit platforms (x86, PowerPC-32) and
  8 for 64-bit platforms (x86-64, PowerPC-64).}

\begin{apient}
 typedef enum { Arch_x86, Arch_x86_64, Arch_ppc32, Arch_ppc64 }
 Architecture;
 virtual Dyninst::Architecture getArchitecture() = 0;
\end{apient}
\apidesc{Return the appropriate architecture for the target
process.}

\begin{apient}
virtual bool getRegValue(Dyninst::MachRegister reg, 
                         Dyninst::THR_ID thread, 
                         Dyninst::MachRegisterVal &val) = 0
\end{apient}
\apidesc{
    This method takes a register name as input, \code{reg}, and returns the
    value in that register in \code{val} in the thread thread.
	
    This method returns \code{true} on success and \code{false} on error.  
}

\begin{apient}
virtual bool readMem(void *dest, Address source, size_t size) = 0
\end{apient}
\apidesc{
    This method reads memory from the target process. Parameter \code{dest} should
    point to an allocated buffer of memory at least \code{size} bytes in the host
    process. Parameter \code{source} should contain an address in the target process to
    be read from. If this method succeeds, \code{size} bytes of memory is copied from
    \code{source}, stored in \code{dest}, and \code{true} is returned. This
    method returns \code{false}
    otherwise.
}

\begin{apient}
virtual bool getThreadIds(std::vector<Dyninst::THR_ID> &threads) = 0
\end{apient}
\apidesc{
    This method returns a list of threads whose call stacks can be walked in the
    target process. Thread are returned in the \code{threads} vector. In some cases,
    such as with the default \code{ProcDebug}, this method returns all of the threads
    in the target process. In other cases, such as with \code{ProcSelf}, this method
    returns only the calling thread. 

    The first thread in the \code{threads} vector (index 0) will be used as the default
    thread if the user requests a stackwalk without specifying an thread (see
    \code{Walker::WalkStack}).
    
    This method returns \code{true} on success and \code{false} on error.  }

\begin{apient}
virtual bool getDefaultThread(Dyninst::THR_ID &default_tid) = 0
\end{apient}
\apidesc{
	This method returns the thread representing the initial process in the
    \code{default\_tid} output parameter.
    
    This method returns \code{true} on success and \code{false} on error.  
}

\begin{apient}
virtual Dyninst::PID getProcessId()
\end{apient}
\apidesc{
    This method returns a process ID for the target process. The default
    \code{ProcessState} implementations (\code{ProcDebug} and \code{ProcSelf}) will return a PID on
    UNIX systems and a HANDLE object on Windows.
}

\begin{apient}
Walker *getWalker() const;
\end{apient}
\apidesc{Return the \code{Walker} associated with the current process
  state. }
   
\begin{apient}
std::string getExecutablePath();
\end{apient}
\apidesc{
Returns the name of the executable associated with the current process state.
}

\paragraph{Class LibraryState}

\definedin{procstate.h}

\code{LibraryState} is a helper class for \code{ProcessState} that provides information about
the current DSOs (libraries and executables) that are loaded into a process'
address space. FrameSteppers frequently use the LibraryState to get the DSO
through which they are attempting to stack walk.

Each \code{Library} is represented using a \code{LibAddrPair} object, which is defined as
follows:

\begin{apient}
typedef std::pair<std::string, Dyninst::Address> LibAddrPair
\end{apient}
\apidesc{
    \code{LibAddrPair.first} refers to the file path of the library that was
    loaded, and \code{LibAddrPair.second} is the load address of that library in
    the process' address space. The load address of a library can be added to a
    symbol offset from the file in order to get the absolute address of a
    symbol.
}

\begin{apient}
virtual bool getLibraryAtAddr(Address addr, LibAddrPair &lib) = 0
\end{apient}
\apidesc{
    This method returns a DSO, using the \code{lib} output parameter, that is
    loaded over address \code{addr} in the current process.

    This method returns \code{false} if no library is loaded over \code{addr} or
    an error occurs, and \code{true} if it successfully found a library.
}

\begin{apient}
virtual bool getLibraries(std::vector<LibAddrPair> &libs) = 0
\end{apient}
\apidesc{
	This method returns all DSOs that are loaded into the process' address space
    in the output vector parameter, \code{libs}. 
    
    This method returns \code{true} on success and \code{false} on error.  
}

\begin{apient}
virtual void notifyOfUpdate() = 0
\end{apient}
\apidesc{
    This method is called by the \code{ProcessState} when it detects a change in
    the process' list of loaded libraries. Implementations of
    \code{LibraryStates} should use this method to refresh their lists of loaded libraries.
}

\begin{apient}
virtual Address getLibTrapAddress() = 0
\end{apient}
\apidesc{
    Some platforms that implement the System/V standard (Linux)
    use a trap event to determine when a process loads a library. A
    trap instruction is inserted into a certain address, and that trap will
    execute whenever the list of loaded libraries change. 
	
    On System/V platforms this method should return the address where a trap
    should be inserted to watch for libraries loading and unloading. The
    ProcessState object will insert a trap at this address and then call
    notifyOfUpdate when that trap triggers.
	
    On non-System/V platforms this method should return 0.  
}

\subsubsection{Class SymbolLookup}
\definedin{symlookup.h}

The \code{SymbolLookup} virtual class is an interface for associating a symbolic
name with a stack frame. Each \code{Frame} object contains an address (the RA)
pointing into the function (or function-like object) that created its stack
frame. However, users do not always want to deal with addresses when symbolic
names are more convenient. This class is an interface for mapping a \code{Frame} object's RA into a name.

In addition to getting a name, this class can also associate an opaque object
(via a \code{void*}) with a Frame object. It is up to the \code{SymbolLookup}
implementation what to return in this opaque object.

The default implementation of \code{SymbolLookup} provided by StackwalkerAPI
uses the \code{SymLite} tool to lookup symbol names. It returns a Symbol
object in the anonymous \code{void*}.

\begin{apient}
SymbolLookup(std::string exec_path = "");
\end{apient}
\apidesc{Constructor for a \code{SymbolLookup} object.}

\begin{apient}
virtual bool lookupAtAddr(Address addr, 
                          string &out_name, 
                          void* &out_value) = 0
\end{apient}
\apidesc{
    This method takes an address, \code{addr}, as input and returns the function name,
    \code{out\_name}, and an opaque value, \code{out\_value}, at that address. Output parameter
    \code{out\_name} should be the name of the function that contains
    \code{addr}. Output
    parameter \code{out\_value} can be any opaque value determined by the
    \code{SymbolLookup}
    implementation. The values returned are used by the \code{Frame::getName} and
    \code{Frame::getObject} functions.

    This method returns \code{true} on success and \code{false} on error.
}

\begin{apient}
virtual Walker *getWalker()
\end{apient}
\apidesc{
    This method returns the \code{Walker} object associated with this
    \code{SymbolLookup}.
}

\begin{apient}
virtual ProcessState *getProcessSate()
\end{apient}
\apidesc{
    This method returns the \code{ProcessState} object associated with this
    \code{SymbolLookup}.
}



